\documentclass[12pt, openany]{report}
\usepackage[utf8]{inputenc}
\usepackage[T1]{fontenc}
\usepackage[a4paper,left=2cm,right=2cm,top=2cm,bottom=2cm]{geometry}
\usepackage[frenchb]{babel}
\usepackage{hyperref}
\usepackage{pslatex}
\usepackage{lmodern}
\usepackage{mathrsfs}
\usepackage{listings}
\usepackage{amsmath,amssymb,amsthm}
\usepackage{wrapfig}
\usepackage[pdftex]{graphicx}
\usepackage{color}
\usepackage{url}
\usepackage{textcomp}
\usepackage{colortbl}
\usepackage[]{xcolor}   
\usepackage{enumitem} %Personaliser les symboles des items
\usepackage{pifont} % Pour plus de symboles pour les items 

\renewcommand\familydefault{lmss} % Joli police 

%************************************

 

%Pour les caractères spéciaux 
\def \Z {\mathbb{Z}}
\def \Zq {\mathbb{Z}_q^*}
\def \G {\mathbb{G}}
\def \C {\mathbb{C}}
\def \E {\mathbb{E}}
\def \F {\mathbb{F}}
\def \N {\mathbb{N}}
\def \R {\mathbb{R}}
\def \Q {\mathbb{Q}}
\def \A {\mathcal{A}}
\def \M {\mathcal{M}}
\def \p {\mathcal{P}}
\def \K {\mathbb{K}}
\def \np {\mathcal{NP}}
\def \O {\mathcal{O}}
\def \D {\mathcal{D}}
%\def \C {\mathcal{C}}
\def \Pr {\mathcal{P}}
\def \Ve {\mathcal{V}}
%\def \g {\mathcal{g}}
\def \I {\{0,1\}}


%*******************************************

 %\addbibresource{sample.bib} %Imports bibliography file


\begin{document}

\section {La constante de Davenport}


Soit $G$ un groupe abélien fini noté additivement. La constante de Davenport notée $D(G)$ est définie comme le plus petit entier $t$ tel que toutes suites $S$ de $G$ de longueur supérieur ou égale à $t$ contienne une sous-suite de somme nulle.

Le problème de déterminer la valeur exacte de cette constante a été introduit dans les année 1960 par P.C. Baayen, H. Davenport et P. Erd\H{o}s.

Si $G=\oplus_{i=1}^{r} C_{n_{i}}$ est la somme directe de groupes cycliques $C_{n_{i}}$, où $n_{i}$ représente l'ordre du groupe cyclique vérifiant $n_{i} \vert n_{i+1}$ et $r$ est le rang du groupe $G$. 

Alors on peut définir $D^{*}(G)$ comme 
$$D^{*}(G)=1+\sum_{i=1}^{r}(n_{i}-1).$$

 on a les deux inégalités suivantes
 
$$ D^{*}(G)=\sum_{i=1}^{r}(n_{i}-1)+1\leq D(G)\leq n_{r}\left(1+log \frac{|G|}{n_{r}}\right).$$

D'où 

$$\forall n \in \N^{*}, \ D(C_{n})=n $$

Jusqu'à la fin des années 1960, il était connu que  \fbox{$D(G)=D^{*}(G)$} dans certains cas (voir ). Ci dessous un récapitulatif des cas possibles pour avoir cette égalité. 
\begin{enumerate}
\item Les $p-$groupes.  
\item Les groupes de rang au plus 2. 
\item Pour le groupe $G^{'} \oplus C_{n}$ avec $G^{'}$ est un $p-$groupe et $D(G^{'}) \le 2 \exp(G^{'}) -1$ et $pgcd(n,\exp(G^{'}))=1$.
\item Les groupes de rang 3 (certains cas). (dans la suite $n \in \N$)
	\begin{enumerate}
	\item Pour le groupe $C_{2}^{2} \oplus C_{2n}$. 
	\item Pour le groupe $C_{3}^{2} \oplus C_{3n}$.
	\item Pour le groupe $C_{3} \oplus C_{3n}^{2}$ tel que $n$ et $6$ sont premiers entre eux. 
	\item Pour le groupe $C_{n_{1}} \oplus C_{n_{2}} \oplus C_{n_{3}m}$ avec $n_{1} \vert n_{2} \vert n_{3}$ et $m \in \N$ (sous les conditions suivantes): 
	
		\begin{itemize}
		\item Si  $D(C_{n_{1}} \oplus C_{n_{2}} \oplus C_{n_{3}})=D^{*}(C_{n_{1}} \oplus C_{n_{2}} \oplus C_{n_{3}}) $
		\item Et si $(n_{1} n_{2}^{2}-2n_{2}-n_{1}-2) \le n_{3} $
		\end{itemize}
	\item Pour le groupe $C_{4}^{2} \oplus C_{4n}$. 
	\item Pour le groupe $C_{6}^{2} \oplus C_{6n}$. 
	\item Les groupes de la forme $C_{n}^{3}$ (sous conditions)
		\begin{enumerate}
		\item Si $n=2 p^{k}$ avec $p$ un nombre premier. 
		\item Si  $n=32^{k}$.
		\end{enumerate}
	\end{enumerate}
\item Le groupe de rang 4 de la forme $C_{2}^{3} \oplus C_{2n}$
\item Pour les groupe de rang $5$ sous la forme  $C_{2}^{4} \oplus C_{2k}$ avec $k$ un nombre pair. 
\end{enumerate}



Il est aussi connu que \fbox{$D(G)=D^{*}(G)+1$} dans les cas suivants: 

\begin{enumerate}
\setcounter{enumi}{6}
\item Pour les groupes sous la forme: $C_{2}^{r-1} \oplus C_{6}$ où $r \in \{5,6,7\}.$
\item Pour le groupe $C_{3}^{3} \oplus C_{6}$. 
\end{enumerate}

Pour le cas où \fbox{$D(G)=D^{*}(G)+2$} on connait la valeur de la constante de Davenport pour le groupe: 

\begin{enumerate}
\setcounter{enumi}{8}
\item $C_{2}^{7} \oplus C_{6}$
\end{enumerate}

On sait aussi que la valeur de la constante de Davenport pour les groupe de la forme $C_{2}^{4} \oplus C_{2k}$ avec $k \geq 70$ est %(voir \cite{CS14}): 
$$
D(C_{2}^{4} \oplus C_{2k}) = \left\{
    \begin{array}{ll}
        2k+4 = D^{*}(C_{2}^{4} \oplus C_{2k})& \mbox{si k est paire} \\
        2k+5 = D^{*}(C_{2}^{4} \oplus C_{2k})+1& \mbox{si k est impaire}
    \end{array}
\right.
$$

%Il est aussi connu que \fbox{$D(G)=D^{*}(G)+1$} dans les cas suivants: 

%\begin{itemize}
%\item Pour les groupes sous la forme: $C_{2}^{r-1} \oplus C_{6}$ où $r \in \{5,6,7\}.$
%\item Pour le groupe $C_{3}^{3} \oplus C_{6}$. 
%\end{itemize}

%Pour le cas où \fbox{$D(G)=D^{*}(G)+2$} on connait la valeur de la constante de Davenport pour le groupe: 

%\begin{itemize}
%\item $C_{2}^{7} \oplus C_{6}$
%\end{itemize}

%On sait aussi que la valeur de la constante de Davenport pour les groupe de la forme $C_{2}^{4} \oplus C_{2k}$ avec $k \geq 70$ est (voir \cite{CS14}): 
%$$
%D(C_{2}^{4} \oplus C_{2k}) = \left\{
  %  \begin{array}{ll}
    %    2k+4 = D^{*}(C_{2}^{4} \oplus C_{2k})& \mbox{si k est paire} \\
      %  2k+5 = D^{*}(C_{2}^{4} \oplus C_{2k})+1& \mbox{si k est impaire}
    %\end{array}
%\right.
%$$
\section{Tableau récapitulatif}\label{umlauts}

%*************************** Je commence mon tableau ici *************************
 \newlength\epaisLigne 
 \newcommand\Gline{\noalign{\global\epaisLigne\arrayrulewidth\global
                              \arrayrulewidth 0.08cm}
		    \hline\noalign{\global\arrayrulewidth\epaisLigne}} 
 \newcommand\Mline{\noalign{\global\epaisLigne\arrayrulewidth\global
                              \arrayrulewidth 0.3}
		    \hline\noalign{\global\arrayrulewidth\epaisLigne}}

\begin{tabular}{|c||c|c|c|c|c|}
\Gline  $|G|$ & $D(G)$& Règle(s) appliquée(s) & Le groupe\\
\Gline  1 & 1 & 2 & $ C_{1}$ \\
\hline  2 & 2 & 2 & $C_{2}$ \\
\hline  3 & 3 & 2 & $C_{3}$ \\  
\hline  4 & 4 & 2 & $C_{4}$ \\ 
\hline  4 & 3 & 1 ou 2 & $C_{2}^{2}$\\
\hline  5 & 5 & 2 & $C_{5}$ \\
\hline  6 & 6 & 2 &  \textcolor{blue}{$C_{2} \oplus C_{3}$ $\simeq C_{6}$} \\
\hline  7 & 7 & 2 & $C_{7}$ \\
\hline  8 & 8 & 2 & $C_{8}$  \\
\hline  8 & 5 & 2 & $ C_{2} \oplus C_{4}$ \\
\hline  8 & 4 & 1 & ${C_{2}}^{3}$ \\
\hline  9 & 9 & 2 & $ C_{9}$ \\
\hline  9 & 5 & 1 et 2& $ C_{3} \oplus C_{3}$\\
\hline  10 & 10 & 2 & $C_{10}$ \\
\hline  10 & 10 & 2 & \textcolor{blue}{$C_{2} \oplus C_{5} \simeq C_{10}$} \\
\hline  11& 10 & 2 & $C_{11}$\\
%\hline  12  & 12 & 12 & $G \simeq C_{12}$ est cyclique \\
\hline  12 & 12 & 2 & \textcolor{blue} {$C_{4} \oplus C_{3} \simeq C_{12}$} \\
\hline  12 & 7 & 2 & \textcolor{blue}{$ C_{3}\oplus C_{2} \oplus C_{2} \simeq C_{6} \oplus C_{2}$}   \\
\hline  13 & 13 & 2 & $C_{13}$\\
%\hline  14 & 14 & 14 & $G \simeq C_{14}$ est cyclique\\
\hline  14 & 14 & 2 & \textcolor{blue}{$C_{7}\oplus C_{2} \simeq C_{14}$} \\
%\hline  15 & 15 & 15 & $G \simeq C_{15}$ est un groupe cyclique  \\
\hline  15 & 15 & 2 & \textcolor{blue}{$C_{5}\oplus C_{3} \simeq C_{15}$} \\
\hline  16  & 16 & 1 ou 2 & $C_{16}=C_{2^{4}}$  \\
%\hline  16 & 5 & 5 & $G \simeq {C_{2}}^{4}$ ($n$ prime power)\\
\hline  16 & 9 & 2 & $C_{2} \oplus C_{8} $ \\
\hline  16  & 7 & 2 & $C_{4}\oplus C_{4}$  \\  
\hline  16  & 6 & $4-(a)$ & $C_{2} \oplus C_{2} \oplus C_{4}$ \\
\hline  17  & 17 & 2 & $C_{17}$  \\
%\hline  18  & 18 & 18 & $G \simeq C_{18}$ est cyclique\\ 
\hline  18  & 18 & 2 & \textcolor{blue}{$C_{2} \oplus C_{9} \simeq C_{18}$} \\
%\hline  18  & 8 & 2 & $C_{3}\oplus C_{6}$ \\
\hline 18 & 8 & 2 & \textcolor {blue}{$G \simeq C_{2}\oplus C_{3}\oplus C_{3} \simeq C_{3}\oplus C_{6}$} \\
\hline  19  & 19 & 2 & $C_{19}$\\  
%\hline  20  & 20 & 20 & $G \simeq C_{20}$ est cyclique \\  
\hline  20  & 20 & 2 & \textcolor{blue}{$C_{4}\oplus C_{5} \simeq C_{20}$}  \\
%\hline  20  & 11 & 11 & $G \simeq C_{2}\oplus C_{10}$ tel que $r\leq 2$\\
\hline  20  & 11 & 2 & \textcolor{blue}{$C_{2}\oplus C_{2}\oplus C_{5} \simeq C_{2}\oplus C_{10}$}  \\
%\hline  21  & 21 & 2 & $C_{21}$\\      
\hline  21  & 21 & 2 & \textcolor{blue}{$C_{3}\oplus C_{7} \simeq C_{21}$} \\
%\hline  22  & 22 & 22 & $G \simeq C_{22}$ est cyclique \\             
\hline  22  & 22 & 2 & \textcolor{blue}{$C_{2}\oplus C_{11} \simeq C_{22}$} \\                               
\hline  23  & 23 & 2 & $C_{23}$ \\
%\hline  24  & 24 & 2 & $C_{24}$  \\
%\hline  24  & 13 & 2 & $G \simeq C_{2}\oplus C_{12}$ tel que $r\leq 2$ \\
\hline  24  & 13 & 2 & \textcolor{blue}{$C_{2}\oplus C_{3}\oplus C_{4} \simeq C_{2} \oplus C_{12}$}  \\
\hline  24  & 24 & 2 & \textcolor{blue}{$ C_{3}\oplus C_{8} \simeq C_{24}$}\\
\hline  24  & 13 & 2 & \textcolor{blue}{$C_{4}\oplus C_{6} \simeq  C_{2} \oplus C_{12}$}\\
%\hline  24  & & &\textcolor{blue}{$G \simeq C_{3}\oplus C_{2} \oplus C_{2} \oplus C_{2}$}\\
\hline  24  & 8 & $4-(a)$ & $C_{2}\oplus C_{2} \oplus C_{6}$\\
\hline  25  & 25 & 2 & $C_{25}$ \\
\hline  25  & 9 & 1 ou 2 & $C_{5}\oplus C_{5}$ \\


\hline 


\end{tabular}

\begin{tabular}{|c||c|c|c|c|c|}
\Gline  $|G|$ & $D(G)$&  Règle(s) appliquée(s) & Le groupe\\
%\hline  26  & 26 & 26 & $G \simeq C_{26}$ est cyclique \\
\Gline  26  & 26 & 2 &  \textcolor{blue}{$C_{2}\oplus C_{13} \simeq C_{26}$}\\
\hline  27  & 27 & 2 & $C_{27}$ \\
\hline  27  & 7 & 1 & $C_{3} \oplus C_{3} \oplus C_{3} $ \\
\hline  27  & 11 & 2 & $C_{3} \oplus C_{9}$ \\
% Il manque ici queleques lignes 
%\hline  28  & 28 & 28 & $G \simeq C_{28}$ est cyclique\\
\hline  28  & 15 & 2 & \textcolor{blue}{$C_{2}\oplus C_{2} \oplus C_{7} \simeq C_{2} \oplus C_{14}$} \\
\hline  28  & 28 & 2 & \textcolor{blue}{$C_{4} \oplus C_{7} \simeq C_{28}$} \\
%\hline  28  & 15 & 2 & $G\simeq C_{2}\oplus C_{2} \oplus C_{7}$ \textcolor{blue}{$\simeq C_{14}\oplus C_{2}$} \\
\hline  29  & 29 & 2 & $C_{29}$\\ 
\hline  30  & 30 & 2 & \textcolor{blue}{$C_{2} \oplus C_{3} \oplus C_{5} \simeq C_{30}$}\\
%\hline  30  & 30 & 30 & $G \simeq C_{2}\oplus C_{3}\oplus C_{5}$ est isomorphe à $C_{30}$\\
\hline  30  & 30 & 2 &\textcolor{blue}{$C_{5} \oplus C_{6} \simeq C_{30}$} \\
\hline  30  & 30 & 2 &\textcolor{blue}{$C_{2} \oplus C_{15} \simeq C_{30}$}\\
\hline  30  & 30 & 2 &\textcolor{blue}{$C_{3} \oplus C_{10} \simeq C_{30}$} \\
\hline  31  & 31 & 2 & $C_{31}$ \\
\hline  32  & 32 & 2 & $C_{32}$ \\
\hline  32  & 32 & 1 & $C_{{2}^{5}}$\\%  ($n$ prime power)
\hline  32  & 11 & 2 &$C_{4} \oplus C_{8}$ \\
\hline  32  & 17 & 2 & $C_{2} \oplus C_{16}$ \\
\hline  32  &7 & 5 &$ C_{2}\oplus C_{2}\oplus C_{2} \oplus C_{4}$ \\
\hline  32  & 10 & $4-(a)$& $ C_{2}\oplus C_{2}\oplus C_{8}$\\
\hline  33  & 33 & 2 & $C_{33}$ \\
%\Gline  34 & 34 & 2&$C_{34}$  \\
\hline  34 & 34 & 2 & \textcolor{blue}{$C_{2} \oplus C_{17} \simeq C_{34}$}\\
%\hline  35 & OUI & 35 & 35 & $G\simeq C_{35}$ est cyclique\\ 
\hline  35 & 35 & 2 & \textcolor{blue}{$C_{5}\oplus C_{7} \simeq C_{35}$} \\
%\hline  36 & OUI & 36 & 36 & $G\simeq C_{36}$ est cyclique\\
%\hline  36 & & & & $G\simeq C_{2}\oplus C_{2}\oplus C_{3}\oplus C_{3}$\\
\hline  36 & 36 & 2 &\textcolor{blue}{$C_{4} \oplus C_{9} \simeq C_{36}$} \\ 
\hline  36 & 14 & 2 & $C_{3} \oplus C_{12}$  \\ 
\hline  36 & 11 & 2 & $C_{6} \oplus C_{6}$  \\ 
%\hline  36 & & & &$G\simeq C_{6}\oplus C_{2}\oplus C_{3}$ \\
%\hline  36 & & & &$G\simeq C_{9}\oplus C_{2}\oplus C_{2}$ \\
%\hline  36 & & & &$G\simeq C_{4}\oplus C_{3}\oplus C_{3}$\\
\hline  36 & 19 & 2 &$ C_{2} \oplus C_{18}$\\
\hline  37 & 37 & 2 & $C_{37}$ \\
%\hline  38 & OUI & 38 & 38 & $G\simeq C_{38}$ est cyclique \\
\hline  38 & 38 & 2 & \textcolor{blue}{$ C_{2}\oplus C_{19} \simeq C_{38}$} \\
%\hline  39& OUI & 39 & 39 & $G\simeq C_{39}$ est cyclique\\
\hline  39& 39 & 2 &\textcolor{blue}{$C_{3}\oplus C_{13} \simeq C_{39}$} \\
%\hline  40& OUI & 40 & 40 & $G\simeq C_{40}$ est cyclique\\
%\hline  40& & & & $G\simeq C_{2}\oplus C_{2}\oplus C_{2}\oplus C_{5}$\\
\hline  40& 40 & 2 & \textcolor{blue}{$C_{5}\oplus C_{8} \simeq C_{40}$} \\
\hline  40& 21 & 2 & \textcolor{blue}{$ C_{4} \oplus C_{10} \simeq C_{2} \oplus C_{20}$} \\
%\hline  40& OUI & 21 & 21 & $G\simeq C_{20}\oplus C_{2}$ tel que $r \leq 2$\\
\hline  40& 21 & 2& \textcolor{blue}{$C_{2}\oplus C_{4}\oplus C_{5} \simeq C_{2} \oplus C_{20}$}\\
\hline  40&12 &$4-(a)$ & $C_{2}\oplus C_{2}\oplus C_{10}$ \\
\hline  41& 41 & 2 & $C_{41}$ \\
%\hline  42& OUI & 42 & 42 & $G\simeq C_{42}$ est cyclique\\
\hline  42& 42 & 2 & \textcolor{blue}{$C_{2}\oplus C_{3}\oplus C_{7} \simeq C_{42}$} \\
\hline  42& 42 & 2 & \textcolor{blue}{$C_{2}\oplus C_{21} \simeq C_{42}$} \\
\hline  42& 42 & 2 & \textcolor{blue}{$C_{3}\oplus C_{14} \simeq C_{42}$} \\
\hline  42& 42 & 2 & \textcolor{blue}{$C_{6}\oplus C_{7} \simeq C_{42}$} \\
\hline  43& 43 & 2& $C_{43}$ \\
\hline  44& 44 & 2 & $C_{44}$ \\
\hline  44& 23 & 2 & \textcolor{blue}{$C_{2}\oplus C_{2}\oplus C_{11} \simeq C_{2} \oplus C_{22}$} \\
\hline  44&  44 & 2 & \textcolor{blue}{$C_{4}\oplus C_{11} \simeq C_{44}$} \\
\hline  44&  23 & 2 & $C_{2} \oplus C_{22}$ \\
\hline  45&  45 & 2 & $C_{45}$ \\
\hline  45& 9 & 2 & $ C_{5}\oplus C_{5}$ \\
%\hline  46&  46 & 46 & $G\simeq C_{46}$ est cyclique \\
\hline  46&  46 & 2 &\textcolor{blue}{$C_{2}\oplus C_{23} \simeq C_{46}$}  \\
\hline  47& 47 & 2 & $C_{47}$ \\
\hline                             
\end{tabular}


\begin{tabular}{|c||c|c|c|c|c|}
\Gline  $|G|$ & $D(G)$&  Règle(s) appliquée(s) & Le groupe\\
\hline  48& 48 & 2 &  \textcolor{blue}{$C_{3}\oplus C_{16} \simeq C_{48}$} \\
\hline  48& 25& 2 & $C_{2} \oplus C_{24}$  \\
\hline  48&15 & 2 &  \textcolor{blue}{$C_{3} \oplus C_{4}\oplus C_{4} \simeq C_{4} \oplus C_{12}$} \\
\hline  48& 25 & 2 & \textcolor{blue}{$C_{6}\oplus C_{8} \simeq C_{2} \oplus C_{24}$}\\
%\hline  48& OUI &15 & 15& $C_{12}\oplus C_{4}$ tel que $r\leq 2$ \\
\hline  48& &  &$G\simeq C_{2}\oplus C_{2}\oplus C_{2}\oplus C_{2}\oplus C_{3} $\\
\hline  48& &  &$G\simeq C_{2}\oplus C_{2}\oplus C_{3}\oplus C_{4}$ \\
\hline  48& 9 & 5 &$G\simeq C_{2}\oplus C_{2}\oplus C_{2}\oplus C_{6}$ \\
%\hline  48& &  &$G\simeq C_{8}\oplus C_{2}\oplus C_{3}$\\
\hline  48&14 & $4-(a)$  &$G\simeq C_{2}\oplus C_{2}\oplus C_{12}$\\
%\hline  48& &  &$G\simeq C_{6}\oplus C_{2}\oplus C_{4}$\\
\hline  49&  49 & 2 &$ C_{49}$ \\
\hline  49& 13 & 1 ou 2 & $C_{7}\oplus C_{7}$ \\
\hline  50& 50 & 2 &$C_{50}$ \\
%\hline  50&  & &$G\simeq C_{2}\oplus C_{5}\oplus C_{5}$\\
\hline  50& 14 & &$G\simeq C_{5}\oplus C_{10}$\\
%\hline  50&  & &$G\simeq C_{25}\oplus C_{2}$\\ 
\hline  51 & 51 & 2 & $ C_{51}$ \\
\hline  52 & 52 & 2 & $ C_{52}$  \\
\hline  52 & 27 & 2 & $C_{2}\oplus C_{26}$ \\
%\hline  52& & & $G\simeq C_{2}\oplus C_{2}\oplus C_{13}$  \\
%\hline  52 & 52 & 52 & $G\simeq C_{4}\oplus C_{13}$ est isomorphe à $C_{4}\oplus C_{13}$\\
\hline  53& 53 & 2 & $ C_{53}$  \\
%\hline  54&  54 & 54 & $G\simeq C_{54}$ est cyclique\\
\hline  54&  54 & 2 & \textcolor{blue}{ $C_{2}\oplus C_{27} \simeq C_{54}$} \\ 
\hline  54&  20 & 2 &  \textcolor{blue}{$ C_{6}\oplus C_{9} \simeq C_{3} \oplus C_{18}$} \\
%\hline  54&  20 & 20 & $G\simeq C_{18}\oplus C_{3}$ tel que $r\leq 2$\\ 
%\hline  54& &  & $G\simeq C_{2}\oplus C_{3}\oplus C_{3}\oplus C_{3} $\\
\hline  54& 10& $4-(b)$ &$ C_{3}\oplus C_{3}\oplus C_{6} $\\
%\hline  54& &  & $G\simeq C_{9}\oplus C_{3}\oplus C_{2}$ \\
%\hline  55&  55 & 55 & $G\simeq C_{55}$ est cyclique \\            
\hline  55&  55 & 2 &  \textcolor{blue}{$C_{5}\oplus C_{11} \simeq C_{55}$}\\
\hline  56& 56 & 2 & $C_{56}$\\
%\hline  56& &  & $G\simeq C_{2}\oplus C_{2}\oplus C_{2}\oplus C_{7}$\\
%\hline  56& &  & $G\simeq C_{4}\oplus C_{2}\oplus C_{7}$\\
\hline  56&16&$4-(a)$& $C_{2}\oplus C_{2}\oplus C_{14}$\\
%\hline  56& &  & $G\simeq C_{4}\oplus C_{14}$\\   
%\hline  56& &  & $G\simeq C_{4}\oplus C_{14}$\\   
\hline  56& 14 &3 & $C_{8}\oplus C_{7}$\\ %D(G^{'}=8)<=15 et (8,7)=1
\hline  56& 29 & 2 & $C_{2}\oplus C_{28}$\\
\hline  57&  57 & 2 & $C_{57}$ \\ 
%\hline  57& &  & $G\simeq C_{19}\oplus C_{3}$\\
\hline  58&  58 & 2 & $C_{58}$ \\
%\hline  58& &  & $G\simeq C_{29}\oplus C_{2}$\\ 
\hline  59&  59 & 2 & $C_{59}$ \\

%\hline 60& 60 & 60 & $G\simeq C_{60}$ \\
\hline 60& 60 & 2 &  \textcolor{blue}{$ C_{4}\oplus C_{3}\oplus C_{5} \simeq C_{60}$} \\ 
%\hline 60& &  &$G\simeq C_{2}\oplus C_{2}\oplus C_{3}\oplus C_{5}$ \\
%\hline 60& &  &$G\simeq C_{15}\oplus C_{2}\oplus C_{2}$ \\
%\hline 60& &  &$G\simeq C_{10}\oplus C_{2}\oplus C_{3}$\\
%\hline 60& &  &$G\simeq C_{6}\oplus C_{2}\oplus C_{5}$ \\   
\hline 60&18&3&$C_{4}\oplus C_{15}$\\%D(G')=4 <=7 et (4,15)=1
%\hline 60& &  &$G\simeq C_{12}\oplus C_{5}$ \\
\hline 60&31&2&$C_{30}\oplus C_{2}$\\
%\hline 60& &  &$G\simeq C_{20}\oplus C_{3}$\\
%\hline 60& &  &$G\simeq C_{6}\oplus C_{10}$\\  
\hline 61& 61& 2 &$C_{61}$\\
%\hline 62& &  &$G\simeq C_{2} \oplus C_{31}$\\
\hline 63& 63& 2 &$C_{63}$ \\ 
%\hline 63& &  &$G\simeq C_{7}\oplus C_{3}\oplus C_{3}$ \\
\hline 63& 15&3 &$C_{7}\oplus C_{9}$ \\ %D(G')=9 
\hline 63& 23 & 2 &$C_{3}\oplus C_{21}$ \\
\hline 64& 64 & 2 &$C_{64}$ \\
\hline 64& &  &$C_{2}\oplus C_{2}\oplus C_{2}\oplus C_{2}\oplus C_{2}$\\
\hline 64& 8& 6 &$C_{2}\oplus C_{2}\oplus C_{2}\oplus C_{2}\oplus C_{4} $\\
\hline 64& 11 &5 &$C_{2}\oplus C_{2}\oplus C_{2}\oplus C_{8}$\\
\hline 64& 18 &$4-(a)$ &$C_{2}\oplus C_{2}\oplus C_{16}$\\
\hline 64& &  &$ C_{2}\oplus C_{2}\oplus C_{4}\oplus C_{4}$\\
\hline 64&19& 2 &$C_{4}\oplus C_{16}$\\
\hline 64& &  &$C_{2} \oplus C_{4}\oplus C_{8}$\\
\hline 64& 10 & $4-(e)$ &$C_{4}\oplus C_{4}\oplus C_{4}$ \\
\hline 65& 65& 2 &$ C_{65}$\\
%\hline 65& &  &$G\simeq C_{5}\oplus C_{13}$\\
\hline 
\end{tabular}

%\section {La constante de Harborth}

%Soit $(G,+,0)$ un groupe abélien fini. La constante de Harborth de $G$, notée $\g(G)$, est le plus petit entier $k$ tel que toute suite d'éléments deux à deux distincts de $G$ de longueur $k$, de manière équivalente tout sous-ensemble de $G$ de cardinal au moins $k$, admet une sous-suite de longueur $\exp(G)$ dont la somme soit nulle. 

%L'étude de Harborth est importante du point de vu géométrique où il considérait une généralisation au treillis. 

%$\g(C_{n})=n$ si n impaire n+1 sinon $C_{n}+C_{n}, C_{2}+C_{2n}$ et C_{3}+C_{3p}$. 







\end{document}



